%Projekt 4
%Ján Jusko
%Typografie a publikování
%2014

\documentclass[a4paper,11pt,titlepage]{article} 
\usepackage[czech]{babel}
\usepackage[utf8]{inputenc}       
\usepackage[left=2cm,text={17cm, 24cm},top=3cm]{geometry}


\bibliographystyle{czplain}
\providecommand{\uv}[1]{\quotedblbase #1\textquotedblleft}

\begin{document}
\begin{titlepage}
\begin{center}
	\Huge
	\textsc{Vysoké učení technické v~Brně\\
	\huge Fakulta informačních technologií}\\
\vspace{\stretch{0.382}}
{ \LARGE Typografie a publikování -- 4. projekt\\
\Huge Bibliografické citace\\}
\vspace{\stretch{0.618}}
\end{center}
{\Large 22. dubna 2014 \hfill
Ján Jusko}
\end{titlepage}


\section{Úvod}

Již od nepaměti bylo lidstvo posedlé získávání, zpracování a šířením informací. Významným milníkem v~historii lidstva je vynalezení knihtisku v~roce 1443 Johanom Gutenbergom, který zviditelnil vědní disciplinu zvanou typografie \cite{Springer:Gutenberg_bible}.
Avšak lidstvo se ve vývoji nezastavilo a s~rozšířením osobních počítačů se zdokonalila i typografie.
\emph{Místo papíru, kovu nebo sázecího stroje je dnes používaná elektronická paměť počítače, v~niž lze vytvořit zdánlivý svět továrny na knihy. Na čem dříve pracoval celý tým pracovníků, to dnes lehce zvládne jeden člověk} \cite{Rybicka:Latex_pro_zacatecniky}.
Pod pojem továrna na knihy má autor citátu na mysli sázecí systém \LaTeX.

\section{Jak vyzrát na \LaTeX}

Při prvním setkání se sázecím systémem \LaTeX \  se uživatel neraz vyděsí. Není třeba se ale ničeho obávat, k~dispozici je mnoho knih v~Českém \cite{Rybicka:Latex_pro_zacatecniky} nebo anglickém \cite{Bringhurst:The_Elements} jazyce i mnoho kvalitně spracovaných webových stránek \cite{Martinek:Latex} \cite{Pecina:Typomil} \cite{Benes:Uvod} zabývajícich se s~touhle problematikou.

\section{Zdokonalte se v~\LaTeX u}

Ty, kterí se odhodlali zdokonalovat v~sázecím systému \LaTeX \  nebo i dokonce pro samotné mistry \LaTeX u budou určitě často narážet na ne zcela zrejmé prvky chování. V~takovým případe jistě ocení různé periodika \cite{Springer:Book_Reviews} které řeší aktuální problémy, nebo dokonce i mezinárodní konference. Na ty se však nemusí být snadné dostat, proto většina z~nich vydáva sborníky \cite{TeX_conf:Proc} které stručně popisují problematiku, která se na dané konferenci vyskytla. 

\section{Přeji hodně zdaru}
Pokud ste se rozhodli pro sázení dokumentů v~systému \LaTeX, věřte nebo ne, přiroste Vám k~srdci. Můžete se tomu věnovat ve volném čase, nebo když ste student tak dokonce i zpracovat bakalářskou \cite{Simek:Transformace_vyrazu} nebo diplomovou \cite{Lebeda:Interpret_jazyka_LaTeX} práci, v~které se můžete problematice věnovat na vyšší úrovni. \break
Přěji hodně zdaru!


\newpage
\bibliography{citace}

\end{document}